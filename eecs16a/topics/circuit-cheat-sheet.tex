\section{Circuit Cheat Sheet}

\begin{frame}{Cheat Sheet: Equations}
    Here are some \textit{useful equations for solving circuits}:\\[1ex]
    \small{
        \begin{tabular}{l l}
            Ohm's Law: & $V = IR$ \\
            Current: & $I = \frac{dQ}{dt}$ \\
            Resistance: & $R = \rho L/A$ \\
            Capacitance: & $C = \epsilon A/d$ \\
            Energy: & $E = \frac{1}{2} CV^2$ \\
            Power: & $P = IV = V^2/R = I^2R$ \\
            Charge-Capacitance: & $Q = CV$ \\
            Resistors in Series: & $R_{eq} = \sum_n R_n$ \\
            Resistors in Parallel: & $1/R_{eq} = \sum_n 1/R_n$ \\
            Capacitors in Series: & $1/C_{eq} = \sum_n 1/C_n$ \\
            Capacitors in Parallel: & $C_{eq} = \sum_n C_n$
        \end{tabular}
    }
\end{frame}

\begin{frame}{Cheat Sheet: Common Circuits}
    \begin{tabular}{m{0.35\textwidth} |m{0.45\textwidth}}
        \textbf{Voltage Divider} & \textbf{Current Divider} \\[0.5ex]
        \hline & \\
        $V_{out} = V_{in} \frac{R_2}{R_1 + R_2}$ & $I_1 = I_{total} \frac{R_2}{R_1 + R_2}$ \\[2ex]
        \begin{circuitikz}[scale=0.85, transform shape]
            \draw (0, 4) node[label={[font=\footnotesize]above:$V_{in}$}] {} to[short, *-] (0.5, 4)
            (0.5, 4) to[R=$R_1$] (0.5, 2)
            (0.5, 2) to[short, -*] (1, 2) node[label={[font=\footnotesize]+30:$V_{out}$}] {}
            (0.5, 2) to[R=$R_2$] (0.5, 0) node[ground] {};
        \end{circuitikz} &
        \begin{circuitikz}[scale=0.85, transform shape]
            \draw (0, 0.5) to[short] (0, 1)
            (0,3) to[V=$V$] (0, 1)
            (0,3) to[short, i=$I_{total}$] (0, 3.5)
            (0, 3.5) to[short] (3.5, 3.5)
            (2, 3.5) to[R=$R_1$, i=$I_1$] (2, 0.5)
            (3.5, 3.5) to[R=$R_2$, i=$I_2$] (3.5, 0.5)
            (0, 0.5) to[short] (3.5, 0.5);
        \end{circuitikz}
    \end{tabular}
\end{frame}
\begin{frame}{Cheat Sheet: Circuit Concepts}
    \begin{itemize}
        \item \textbf{Charge on a capacitor} after $t$ seconds at constant current:
        \begin{align*}
            Q_{total} = It
        \end{align*}
        \item \textbf{Conservation of Charge}:
        \begin{align*}
            Q_{total} = Q_{initial}
        \end{align*}
        \item \textbf{Thevenin/Norton} equivalent circuits:
        \begin{align*}
            V_{Th} = V_{OC} \\
            I_{N} = I_{SC} \\
            R_{eq} = V_{Th} / I_N = V_{OC} / I_{SC}
        \end{align*}
        \item \textbf{Golden Rules} (apply to ideal op amps):
        \begin{align*}
            I_{-} = I_{+} = 0 \, (\text{always}) \\
            V_{-} = V_{+} \, (\text{in negative feedback})
        \end{align*} 
    \end{itemize}
\end{frame}

\begin{frame}{Cheat Sheet: Op Amp Configurations}
    \begin{tabular}{c | c}
        \textbf{Non-Inverting Amplifier} & \textbf{Inverting Amplifier} \\[0.5ex]
        \hline & \\
        $V_{out} / V_{in} = \frac{R_1 + R_2}{R_1}$ & $V_{out} / V_{in} = -R_2/R_1$ \\[2ex]
        \begin{circuitikz}[scale=0.85, transform shape]
            \draw (2.5, 1.5) node[op amp, yscale=-1] (opamp) {}
            (1, 2) node[label={[font=\footnotesize]above:$V_{in}$}] {} to[short, o-] (opamp.+)
            (-0.5, -0.1) node[ground] {} to[short] (-0.5, 0)
            (-0.5, 0) to[R=$R_1$] (1, 0)
            (1, 0) to[R=$R_2$] (3.7, 0)
            (opamp.-) to[short] (1.3, 0)
            (opamp.out) to[short] (3.7, 0)
            (opamp.out) to[short, -o] (4, 1.5) node[label={[font=\footnotesize]above:$V_{out}$}] {};
        \end{circuitikz} &
        \begin{circuitikz}[scale=0.85, transform shape]
            \draw (2.5, 1.5) node[op amp] (opamp) {}
            (-0.4, 2) node[label={[font=\footnotesize]above:$V_{in}$}] {} to[R=$R_1$, o-] (opamp.-)
            (opamp.-) to[short] (1.3, 3)
            (1.3, 3) to[R=$R_2$] (3.7, 3)
            (3.7, 3) to[short] (opamp.out)
            (1, 0.9) node[ground] {} to[short] (1, 1)
            (1, 1) to[short] (opamp.+)
            (opamp.out) to[short, -o] (4, 1.5) node[label={[font=\footnotesize]above:$V_{out}$}] {};
        \end{circuitikz}
    \end{tabular}
\end{frame}