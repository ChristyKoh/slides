\section{Charge Sharing}

\begin{frame}{Charge Sharing}
    Capacitors can be first charged up, then \textit{reconfigured into a different circuit}, usually via switches. \\[5pt]
    \underline{States to Analyze}:
    \begin{enumerate}
        \item \textbf{Initial state}: after charging up
        \item \textbf{Final state}: after charges redistribute in new configuration
    \end{enumerate}

    \textbf{NOTE!} \textit{Charges are *always* conserved from phase 1 to 2.}
\end{frame}

\begin{frame}{Charge Sharing: Steps to Solve}
    \begin{itemize}
        \item\textbf{ Note how many states} you will have to find charges for, and \textbf{draw their equivalent circuits}. Generally, there are two: an \textit{initial state and a final state}. 
        \begin{itemize}
            \item Sometimes, there will be intermediate stages--but you \textit{solve those much like you will a two-state problem}.
        \end{itemize}
        \item Find the \textbf{charges on all capacitors in your initial state} in terms of your knowns.
        \begin{itemize}
            \item Most often, you’ll be given an \textit{initial charge on one cap, or a voltage source, and the capacitances of all caps}. 
            \item For 16A, this is generally taken once the charges stabilize.
        \end{itemize}
        \item Find the \textbf{charges on all capacitors in your second state}, knowing that $Q_{final, total} = Q_{init, total}$ in your system because \textbf{charge is conserved}, in terms of your knowns.
    \end{itemize}
\end{frame}

\begin{frame}{Practice: Charge Sharing}
    \begin{tabular}{m{0.55\textwidth} m{0.35\textwidth}}
        & \multirow{2}{*}{
            \begin{circuitikz}[scale=0.7, transform shape]
                \draw (0, 0) to[C=$C$, v_>=$\,\,V_i$] (0, 3)
                (0, 0) to[short] (3, 0)
                (0, 3) to[nos] (3, 3)
                (3, 0) to[C=$C$, v_>=$\,\,0\,V$] (3, 3);
            \end{circuitikz}
        } \\[5pt]
        Suppose the left capacitor has an initial voltage of $V_i$ & \\[30pt]
    \end{tabular}
    \textbf{What happens when we close the switch?}
\end{frame}

\begin{frame}{Practice: Charge Sharing [Solution]}
    \color{blue}
    \begin{tabular}{m{0.7\textwidth} m{0.2\textwidth}}
        Assuming ideal switch connection, \textbf{total charge must be conserved}. & \multirow{2}{*}{
            \color{black}
            \begin{circuitikz}[scale=0.6, transform shape]
                \draw (0, 0) to[C=$C$, v_>=$\,\,V_i$] (0, 3)
                (0, 0) to[short] (3, 0)
                (0, 3) to[nos] (3, 3)
                (3, 0) to[C=$C$, v_>=$\,\,0\,V$] (3, 3);
            \end{circuitikz}
        } \\[15pt]
        Phase 1 is when switch is open and Phase 2 is after we \textit{close the switch and reach steady state}. & \\[20pt]
    \end{tabular}
    $Q_{1, final} = Q_1 / 2$ because each capacitor has the same voltage and therefore the same charge in Phase 2. \\[10pt]
    However, energy at Phase 2 = $\frac{CV^2}{2} = \frac{CQ^2}{2C^2} = \frac{Q^2}{2C}$. This is half the energy in Phase 1, so energy was dissipated between Phase 1 nd Phase 2!
\end{frame}

\begin{frame}{Challenge Practice: Charge Sharing}
    Phase 1: all $\phi_1$ switches are \textbf{closed}, and all $\phi_2$ switches are \textbf{open}. \\[5pt]
    Phase 2: all $\phi_1$ switches are \textbf{open}, and all $\phi_2$ switches are \textbf{closed}. \\[5pt]
    What is $V_x$ during phase 2 if $C_1 = C$ and $C_2 = 9C$? 
    \begin{center}
        \begin{circuitikz}[scale=0.75, transform shape]
            \draw (0, 3) to[V=$V_{source}$] (0, 0.5) node[ground] {}
            (0, 3) to[short] (2, 3)
            (2, 3) to[ospst, l=$\phi_1$] (4, 3)
            (2, 3) to[short] (2, 1)
            (2, 1) to[cspst, l=$\phi_2$] (4, 1)
            (4, 3) to[C=$C_1$] (4, 1)
            (4, 3) to[cspst, l=$\phi_2$] (6, 3)
            (4, 1) to[ospst, l=$\phi_1$] (6, 1)
            (6, 3) to[C=$C_2$] (6, 1) node[ground] {}
            (6, 3) to[short] (8, 3) node[label={[font=\footnotesize] above:$V_x$}] {}
            (8, 3) to[ospst, l=$\phi_1$] (8, 1) node[ground] {};
        \end{circuitikz}
    \end{center}
\end{frame}

\begin{frame}{Challenge Practice: Charge Sharing [Solution]}
    \color{blue}
    \underline{Phase 1}
    \begin{align*}
        Q_1 = C_1 V_1 = C_1(V_S - 0) = C_1 V_S \\
        Q_2 = C_2 V_2 = (9C)(0 - 0) = 0
    \end{align*}
    \begin{center}
        \color{black}
        \begin{circuitikz}[scale=0.75, transform shape]
            \draw (0, 3) to[V=$V_{source}$] (0, 0.5) node[ground] {}
            (0, 3) to[short] (2, 3)
            (2, 3) to[short] (4, 3)
            (4, 3) to[C=$C_1$, v=$\,$] (4, 1)
            (4, 1) to[short] (6, 1)
            (6, 3) to[C=$C_2$, v=$\,$] (6, 1) node[ground] {}
            (6, 3) to[short] (8, 3) node[label={[font=\footnotesize] above:$V_x$}] {}
            (8, 3) to[short] (8, 1) node[ground] {};
        \end{circuitikz}
    \end{center}
\end{frame}

\begin{frame}{Challenge Practice: Charge Sharing [Solution]}
    \color{blue}
    \underline{Phase 2}
    \begin{align*}
        V_1 = V_x - V_S \,\, \rightarrow \,\, Q_1 = C_1(V_x - V_S) = C(V_X - V_S) \\
        V_2 = V_X \,\, \rightarrow \,\, Q_2 = C_2 V_X = 9CV_X \qquad \qquad \qquad
    \end{align*}
    \begin{center}
        \color{black}
        \begin{circuitikz}[scale=0.75, transform shape]
            \draw (0, 3) to[V=$V_{source}$] (0, 0.5) node[ground] {}
            (0, 3) to[short] (2, 3)
            (2, 3) to[short] (2, 1)
            (2, 1) to[short] (4, 1)
            (4, 3) to[C=$C_1$, v=$\,$] (4, 1)
            (4, 3) to[short] (6, 3)
            (6, 3) to[C=$C_2$, v=$\,$] (6, 1) node[ground] {}
            (6, 3) to[short] (8, 3) node[label={[font=\footnotesize] above:$V_x$}] {};
        \end{circuitikz}
    \end{center}
\end{frame}

\begin{frame}{Challenge Practice: Charge Sharing [Solution]}
    \color{blue}
    Total charge in Phase 1 = Total charge in Phase 2
    \begin{align*}
        Q_{1, phase1} + Q_{2, phase1} = Q_{1, phase2} + Q_{2, phase2} \\
        CV_ + 0 = C(V_X - V_S) + 9CV_X \\
        2CV_S = 10CV_X \,\, \rightarrow \,\, V_X = V_S / 5
    \end{align*}

    \begin{center}
        \color{black}
        \begin{circuitikz}[scale=0.75, transform shape]
            \draw (0, 3) to[V=$V_{source}$] (0, 0.5) node[ground] {}
            (0, 3) to[short] (2, 3)
            (2, 3) to[short] (2, 1)
            (2, 1) to[short] (4, 1)
            (4, 3) to[C=$C_1$, v=$\,$] (4, 1)
            (4, 3) to[short] (6, 3)
            (6, 3) to[C=$C_2$, v=$\,$] (6, 1) node[ground] {}
            (6, 3) to[short] (8, 3) node[label={[font=\footnotesize] above:$V_x$}] {};
        \end{circuitikz}
    \end{center}
\end{frame}