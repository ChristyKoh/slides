\section{Diagonalization}

\begin{frame}{Diagonalization}
    \textbf{Goal}:\\[-1ex]
    \begin{itemize}
        \item Find a basis such that a given matrix transformation can be expressed through \textbf{scaling by eigenvalues}. This is also called \textbf{eigendecomposition}.
        \begin{align*}
            A = P \Lambda P^{-1}
        \end{align*}
        \item Motivation: makes computing powers of $A$ easier.
    \end{itemize}
    \textbf{How}:\\[-1ex]
    \begin{itemize}
        \item $\Lambda$ is a \textit{diagonal matrix of eigenvalues}.
        \item $P$ is a matrix whose \textit{columns are the corresponding eigenvectors}.
    \end{itemize}
\end{frame}

\begin{frame}{Practice: Diagonalization}
    Diagonalize the following matrix:
    \begin{align*}
        A = \begin{bmatrix}
            3 & 3 \\
            1 & 5
        \end{bmatrix}
    \end{align*}
\end{frame}

\begin{frame}{Practice: Diagonalization [Solution]}
    \color{blue}{
        First, \textbf{find eigenvalues}.
        \begin{align*}
            \det \Bigg(\begin{bmatrix}
                3 & 3 \\
                1 & 5
            \end{bmatrix} - \lambda I \Bigg)
            = \det\Bigg(\begin{bmatrix}
                3 - \lambda & 3 \\
                1 & 5 - \lambda
            \end{bmatrix} = 0
            \Bigg) \\
            (3 - \lambda)(5  \lambda) - 3 = (\lambda - 2)(\lambda - 6) = 0
            \lambda = 2, 6 \\[2ex]
            \Lambda = \begin{bmatrix}
                2 & 0 \\
                0 & 6
            \end{bmatrix}
        \end{align*}
    }
\end{frame}

\begin{frame}{Practice: Diagonalization [Solution]}
    \color{blue}{
        Next, \textbf{find eigenvectors}.
        \begin{align*}
            A - 2I = 0 \rightarrow 
            \begin{bmatrix}[c c | c]
                1 & 3 & 0\\
                1 & 3 & 0
            \end{bmatrix} \xrightarrow[]{\text{row reduce}}
            \begin{bmatrix}[c c | c]
                1 & 3 & 0\\
                0 & 0 & 0
            \end{bmatrix} \xrightarrow[]{\text{eigenvector}}
            \begin{bmatrix}
                -3 \\ 1
            \end{bmatrix} \\[2ex]
            A - 6I = 0 \rightarrow 
            \begin{bmatrix}[c c | c]
                -3 & 3 & 0\\
                1 & -1 & 0
            \end{bmatrix} \xrightarrow[]{\text{row reduce}}
            \begin{bmatrix}[c c | c]
                -1 & 1 & 0\\
                0 & 0 & 0
            \end{bmatrix} \xrightarrow[]{\text{eigenvector}}
            \begin{bmatrix}
                1 \\ 1
            \end{bmatrix} \\[2ex]
            P = \begin{bmatrix}
                -3 & 1 \\
                1 & 1
            \end{bmatrix}
        \end{align*}
    }
\end{frame}

\begin{frame}{Practice: Diagonalization [Solution]}
    \color{blue}{
        Find the \textbf{inverse of eigenvector matrix}.
        \begin{align*}
            P^{-1} = \begin{bmatrix}
                -3 & 1 \\
                1 & 1
            \end{bmatrix}^{-1} = \frac{1}{(-3)(1) - (1)(1)}
            \begin{bmatrix}
                1 & -1 \\
                -1 & -3
            \end{bmatrix} = 
            \begin{bmatrix}
                -1/4 & 1/4 \\
                1/4 & 3/4
            \end{bmatrix}
        \end{align*}
    }
\end{frame}

\begin{frame}{Practice: Diagonalization [Solution]}
    \color{blue}{
        Plug into the diagonalization equation!
        \begin{align*}
            A = P\Lambda P^{-1} \\
            A = \begin{bmatrix}
                -3 & 1 \\
                1 & 1
            \end{bmatrix}
            \begin{bmatrix}
                2 & 0 \\
                0 & 6
            \end{bmatrix} 
            \begin{bmatrix}
                -1/4 & 1/4 \\
                1/4 & 3/4
            \end{bmatrix}
        \end{align*}
    }
\end{frame}